\chapter{Resultados e Conclusão}

\section{Resultados}


\subsection{Marx e o lixo}

Sem deméritos da profissão de catador, ao contrário, bendito seja o lixo da rica burguesia que gera renda para os menos afortunados. Triste pensar que a falta de emprego possa gerar a situação de centenas de pessoas, as vezes jovens e crianças, a depender das migalhas putrefatas do lixo para conseguir uns trocados e alimentação.

E, refletindo sobre o que disse \cite{Balieiro2014}, a cerca da ideologia:
\begin{citacao}
    É a ideologia que nos “apresenta” a realidade. Não podemos apreendê-la senão
    por ela, essa imagem invertida, inversa do que ela é: uma falsa imagem que produz
    uma falsa consciência a respeito das próprias ideias e das relações concretamente
    estabelecidas. Os burgueses dos séculos XV e XVI, por exemplo, não explicitam
    que, ao financiar obras artísticas que colocam o homem como centro da explicação
    para as coisas do mundo, eles estão, na verdade, querendo minar o poder político
    da Igreja e tomar o seu lugar como origem dos saberes e das decisões políticas
    e econômicas. As imagens, que até então eram “chapadas”, ganham contorno e
    perspectiva, as figuras humanas aparecem cada vez mais “humanizadas”. E a ideologia
    não nos revela que isso ocorre para que a ideia de divindade seja suprimida,
    já que mesmo as figuras bíblicas aparecem-nos com aspectos intrinsecamente humanos;
    basta admirarmos uma pintura de Caravaggio.
\end{citacao}

O quando subvertido devemos estar para acompanhar o ideal coletivo, impregnado da realidade imposta pela ideologia dominante? Fica a pergunta para o caro leitor dizer a resposta. Ademais, confesso que as ideias me bombardeiam e merecem mais reflexão.


\section{Conclusão}
